%% Generated file to plot data from a datafile using pgfplots

%================================================================================
%|                                                                              |
%| If you plot a huge amount of data, and TeX runs out on its main memory,      |
%| you have two options that are suggested by pgfplots' manual:                 |
%|  1. You could try lualatex instead of pdflatex                               |
%|  2. You could modify some of latex's settings in the texmf.cnf. How to do    |
%|     for your LaTeX distrobution, Google is your friend.                      |
%|                                                                              |
%| The documentclass 'standalone' is used to crop white space, if that is not   |
%| working for you, you can use article as your document class, and  uncomment  |
%| the 3 lines for using the package 'tightpage'. As a last resort, you can     |
%| always use the command-line tool 'pdfcrop'.                                  |
%| The 'Preview package is used to remove white space around the tikzpicture    |
%| environment. If this fails for you, and/or alternatively you can comment     |
%| the relevant lines just above the '\begin{document}' command                 |
%| and make use of the external tool pdfcrop.                                   |
%|                                                                              |
%================================================================================

\documentclass[11pt]{standalone}
%\documentclass[11pt]{article}

% Set page legths (uncomment for using article and tightpage/pdfcrop):
%\special{papersize=50cm,50cm}
%\hoffset-0.8in
%\voffset-0.8in
%\setlength{\paperwidth}{50cm}
%\setlength{\paperheight}{50cm}
%\setlength{\textwidth}{45cm}
%\setlength{\textheight}{45cm}
%\topskip0cm
%\setlength{\headheight}{0cm}
%\setlength{\headsep}{0cm}
%\setlength{\topmargin}{0cm}
%\setlength{\oddsidemargin}{0cm}
% set the pagestyle to empty (removing pagenumber etc)
\pagestyle{empty}

% load packages:
\usepackage{amsmath}
\usepackage{amssymb}
\usepackage{amstext}
\usepackage{amsfonts}
\usepackage{textcomp}
\usepackage{units}

\usepackage{tikz}
\usepackage{pgfplots}
\usetikzlibrary{spy}
%\usepgfplotslibrary{external}
%\tikzexternalize

% Use newest spacing options (from v. 1.3 on)
\pgfplotsset{compat=1.10}
\pgfplotsset{width=6cm} % orig: 5
\pgfplotsset{height=3.6cm} % orig: 3

\pgfplotsset{grid style={solid}}

% Removing white space by using 'standalone' as the document class, uncomment the following as an alternative:
% Remove white space from generated pdf,
% thus otaining a pdf with only the picture that can
% easily be included in a(nother) tex-document via the usual \includegraphic command.
% Benefit: 1. you can keep your pictures organised in a subfolder and
%          2. the picture remains a vector graphic :)
%\usepackage[active, tightpage]{preview}
%\PreviewEnvironment{tikzpicture}
%\setlength\PreviewBorder{0pt}
% Alternatively, delete the three lines above and run 'pdfcrop filename.pdf',
% the result should be the same.

\begin{document}

\centering

\begin{tikzpicture}[spy using outlines={circle, magnification=4, connect spies}]
  \begin{axis}[width=6cm, height=3.6cm,
        axis background/.style={fill=gray!20},
        axis x line*=bottom, axis y line*=left,
        %xmode=log, % logarithmic x axis
        %log basis x=, % log basis, if empty the natural logarithm is used
        % If more precision on x or y axis is needed, uncomment the relevant lines below:
        % scaled x ticks=false,
        % x tick label style={/pgf/number format/fixed, /pgf/number format/precision=3}, % 3 for 3 floating point digits
        % scaled y ticks=false,
        % y tick label style={/pgf/number format/fixed, /pgf/number format/precision=3}, % 3 for 3 floating point digits
        x tick label style={font=\tiny},
        y tick label style={font=\tiny},
        scale only axis, % might get 'dimension too large' error if switched on
        %minor tick num=0,
        xmin=0,
        %ymin=0, ymax=30,
        % restrict x to domain=-10:10, % use this if you get a 'dimension too large' error
        % restrict y to domain=-10:10, % use this if you get a 'dimension too large' error
        xlabel={$x$ Coordinate},
        ylabel={$u_x [ms^{-1}]$},
        label style={font=\tiny},
        %x dir=reverse,
        legend cell align=left, % best if aligned left
        legend style={legend pos=south east,
                      % specify legend entries:
                      % example: legend entries={entry1, entry2, entry3},
                      % if legend entries are too long, specify max text width/depth:
                      % nodes={text width=30pt,text depth=40},
                      % if you want to put the legend outside the figure envirmonent, do:
                      % legend to name=legendlabel,
                      % and then '\ref{legendlabel}' where you want the legend to appear
                      % don't forget to run pdflatex twice for it to pick up the changed reference!
                      legend columns=1,
                      %legend entries={Grandy, bounded, void, bounded dg, void dg},
                      legend entries={$\text{Gl}_{\text{b,P1-P1}}$, $\text{Gl}_{\text{b,P1}_{\text{DG}}\text{-P2}}$},
                      font=\tiny},
        grid=both
        ]
    \addplot[color=magenta, solid, line join=round] table[x=Points0, y=Velocity0] {./data/bounded_x_0-1_interp_0-0001.csv};
    \addplot[color=blue, dashed, line join=round] table[x=Points0, y=Velocity0] {./data/boundeddg_x_0-1_interp_0-0001.csv};
    % highlighting where the cylinder is:
    \draw[color=red, fill=red, opacity=0.35] (axis cs:0.15,-0.025) rectangle (axis cs:0.25,0.3);
    % for the spy:
    \coordinate (spypoint) at (axis cs:0.3,-0.0035);
    \coordinate (spyviewer) at (axis cs:0.85,0.09);
    \begin{scope}
      \spy[red,size=1.2cm] on (spypoint) in node at (spyviewer);
    \end{scope}
  \end{axis}
\end{tikzpicture}

\end{document}
