% header.tex
\documentclass[
    a4paper,
    12pt,
    %oneside,
    twoside,
    %bibtotoc,	% Bibliography to table of contents
    onecolumn,	% write in one column
    %idxtotoc,	% index to toc
    %abstracton
    %draft
]{report}

% definition layout and print space
\setlength{\hoffset}{-1in}
\setlength{\voffset}{-0.65in}%-0.5in
%\setlength{\voffset}{-0.5in}
% \setlength{\topmargin}{2.5cm}
% \setlength{\oddsidemargin}{4cm} % normally using this as margin
\setlength{\oddsidemargin}{3cm} % reducing to 3cm for IC thesis
\setlength{\evensidemargin}{\oddsidemargin} % reducing to 3cm for IC thesis
% \setlength{\textwidth}{16cm}
%\marginparsep = 11pt
%\marginparwidth = 54pt
%\fancyheadoffset[LE,RO]{\marginparsep+\marginparwidth}
% \setlength{\textwidth}{14.75cm}
%\setlength{\textwidth}{14.25cm} % normally using this
% IC rules: content should be centered, thus pagewidth=21cm - 2*3cm (margins) = 15cm of text:
\setlength{\textwidth}{15cm}
% \setlength{\textheight}{23,4cm}
\setlength{\textheight}{23cm}
% \setlength{\textheight}{21.4cm}
\setlength{\headsep}{0.75cm}
\setlength{\footskip}{30pt}
\setlength{\headheight}{28pt}
\setlength{\parindent}{0pt} % no indentation 
\setlength{\parskip}{1.5ex} % vertical space between paragraphs
%\setlength{\baselineskip}{1.5\baselineskip} % vertical space between lines (copy to where needed) 
%\setlength{\textfloatsep}{5.5\baselineskip} % distance between floats on the top or the bottom and the text;
\setlength{\floatsep}{\baselineskip} % distance between two floats;
\setlength{\intextsep}{1.5\baselineskip} % distance between floats inserted inside the page text (using h) and the text proper.
%\setlength{\belowcaptionskip}{1.5\baselineskip}
\usepackage{varwidth}

% Setting header/footer with fancyheaders
% fancyheaders.tex

% New commands for loading own pagestyles
\newcommand\mypagestyle[1]{\pagestyle{#1}}
\newcommand\mylisthead[1]{\fancyhead[LE,RO]{\textsl{#1}}}

% using fancyhdr to define header and footer
\usepackage{fancyhdr}
\pagestyle{fancy}
\renewcommand\sectionmark[1]{\markright{\textsl{\thesection: #1}}}
\renewcommand\chaptermark[1]{\markboth{\textsl{\chaptername\ \thechapter: #1}}{}}

% myfancy
\fancypagestyle{myfancy}{
\fancyhead{}
\fancyhead[LE]{\leftmark}
\fancyhead[RO]{\rightmark}
\fancyfoot{}
\fancyfoot[C]{}
\fancyfoot[RO,LE]{\thepage}
\renewcommand{\headrulewidth}{0.4pt}
\renewcommand{\footrulewidth}{0.4pt}}

% plain
% redefine the style 'plain', 
% 'plain' is used when calling \chapter{foo}
\fancypagestyle{plain}{
\fancyhead{}
\fancyfoot{}
\fancyfoot[LE,RO]{\thepage}
\renewcommand{\headrulewidth}{0.0pt}
\renewcommand{\footrulewidth}{0.4pt}}

% Lists, e.g. for table of content, list of figures etc.
\fancypagestyle{lists}{
\fancyhead{}
\fancyfoot{}
\fancyfoot[LE,RO]{\thepage}
\renewcommand{\headrulewidth}{0.4pt}
\renewcommand{\footrulewidth}{0.4pt}}

% References
\fancypagestyle{myreferences}{
\mypagestyle{myfancy}
\fancyhead{}
\fancyhead[LE,RO]{\textsl{\bibname}}
\renewcommand{\headrulewidth}{0.4pt}
\renewcommand{\footrulewidth}{0.4pt}}

%
% END
%

\input{preamble/laodlistings}
% Implementing Abbreviation
% \input{abbreviation/build_abbreviation}
% % adding a nomenclature
\usepackage{nomencl}
\makenomenclature
% rename Nomenclature:
%\renewcommand\nomname{Abbreviations}

% defining entries of the nomenclature
% math symbols
\nomenclature[aa]{$p$}{Pressure}
\nomenclature[ab]{$\boldsymbol{u_f}$}{Fluid velocity vector}
\nomenclature[ab]{$\boldsymbol{u_s}$}{Solid velocity vector}
\nomenclature[ab]{$\mathbf{u_f}$}{Discrete fluid velocity vector}

% mesh and elements
\newcommand\MeshF{\mathcal{T}_F}
\newcommand\MeshS{\mathcal{T}_S}
\newcommand\NF{\mathcal{N}_F}
\newcommand\NS{\mathcal{N}_S}
\newcommand\KF[1][]{\mathcal{K}_{F_{#1}}\xspace}
\newcommand\KS[1][]{\mathcal{K}_{S_{#1}}\xspace}
\nomenclature[na]{$\MeshF$}{Fluid mesh}
\nomenclature[na]{$\MeshS$}{Solid mesh}
\nomenclature[nb]{$\NF$}{Number of basis functions on the fluid mesh ($\MeshF$)}
\nomenclature[nb]{$\NS$}{Number of basis functions on the solid mesh ($\MeshS$)}
\nomenclature[nb]{$\KF$}{Element on the fluid mesh ($\MeshF$)}
\nomenclature[nb]{$\KS$}{Element on the solid mesh ($\MeshS$)}

% abbreviations
\newcommand\projgrandy{$\text{Gr}_{\text{\popo}}$\xspace}
\newcommand\projbnd{$\text{Gl}_{\text{b,\popo}}$\xspace}
\nomenclature[z]{\projgrandy}{Grandy interpolation}
\nomenclature[z]{\projbnd}{Bounded Galerkin projection}
\nomenclature[z]{CFD}{Computational Fluid Dynamics}
\nomenclature[z]{FEM}{Finite Element Method}
\nomenclature[z]{FV}{Finite Volume}
\nomenclature[z]{FSI}{Fluid-Solid Interaction}
\nomenclature[z]{CPU}{Central Processing Unit}

% defining subgroups of the nomenclature
\renewcommand{\nomgroup}[1]{%
  \ifthenelse{\equal{#1}{A}}{\item[\textbf{Variables}]}{%
  \ifthenelse{\equal{#1}{M}}{\item[\textbf{Math symbols}]}{%
  \ifthenelse{\equal{#1}{N}}{\item[\textbf{Mesh related symbols}]}{%
  \ifthenelse{\equal{#1}{Z}}{\item[\textbf{Abbreviations}]}{%
  \ifthenelse{\equal{#1}{G}}{\item[\textbf{Constants}]}{}}}}}}


% Define the layout of chapter titles:
\usepackage[Bjarne]{fncychap}% Bjarne, Sonny, Lenny, Bjornstrup
% \ChNameUpperCase \ChNameVar{\raggedleft\normalsize\rm} \ChRuleWidth{0.5pt}
\ChNameAsIs \ChNameVar{\raggedleft\Huge\rm} \ChRuleWidth{0.5pt}
\ChNumVar{\raggedleft\Huge}
\ChNameAsIs \ChNameVar{\raggedleft\LARGE\rm} \ChRuleWidth{0.5pt}
\ChNumVar{\raggedleft\rm\bfseries\LARGE}

% \ChTitleUpperCase
\ChTitleAsIs
\ChTitleVar{\raggedleft\rm\bfseries\Huge}
%\ChNameVar{\raggedleft\Large} \ChNumVar{\raggedleft\Large} \ChTitleVar{\raggedleft\Huge}
%\ChRuleWidth{0.5pt} \ChNameUpperCase
%\ChTitleAsIs
%
% \usepackage{changepage}
% \changepage{textheight}{textwidth} % Change layout in middle of document
%  {evensidemargin}{oddsidemargin}%
%  {columnsep}{topmargin}%
%  {headheight}{headsep}%
%  {footskip}

\usepackage{calc}

\usepackage{setspace} % \doublespacing, \onehalfspacing, \singlespacing

% Formats
%\usepackage{geometry}
%\geometry{left=3.5cm,textwidth=14cm,top=1.5cm,textheight=23cm}
%\usepackage[sf]{titlesec} % Headers in Sans Serif
%\usepackage{ucs} % UTF-8
\usepackage{textcomp}

% Numbering of headers:
 \usepackage[nottoc,notlof,notlot]{tocbibind}	% Adding all titles to the toc automatically
\setcounter{secnumdepth}{4}	% x+1 section numbering
\setcounter{tocdepth}{2}	% x+1 section numbering in the table of content (toc)
%\pagenumbering{arabic}		% options are: % arabic; roman; Roman; alph; Alph
\usepackage{afterpage}
\usepackage{nextpage}


% ifthen
\usepackage{ifthen}
\usepackage{shortvrb}


% UK-English
\usepackage[english]{babel} 
\usepackage[babel]{csquotes} % correct quotation marks, e.g. \enquote{some text}; alternatively \glqq und \grqq

% Math packages
\usepackage[intlimits,sumlimits]{amsmath} % setting placements of limits
\usepackage{amssymb}
\usepackage{amstext}	
\usepackage{amsfonts}
% $\mathbf{N} + \mathrm{N} + \mathcal{N} + \mathsf{N} + \mathtt{N} + \mathit{N} + \mathbf{N} + \mathbb{N} + \mathfrak{N} + \mathscr{N}$
\usepackage{mathtools} % dynamic behaviour of amsmath
\usepackage{mathrsfs} % provides rsfs fonts (i.e. for Laplace fcn)
\usepackage{amsthm}
\usepackage{units} % package for unit, e.g. $\unit[\frac{1}{8}]{m}$
\usepackage{siunitx}
%\sisetup{output-exponent-marker=\textsc{e}}
%\sisetup{output-exponent-marker=\textsc{e}, bracket-negative-numbers, open-bracket={\text{-}}, close-bracket={}}
%\sisetup{output-exponent-marker=10, bracket-negative-numbers, open-bracket={\text{-}}, close-bracket={}}

% TikZ:
\usepackage{tikz}
\usetikzlibrary{shapes.geometric, arrows, intersections, through}
\usetikzlibrary{decorations.text, decorations.shapes, backgrounds}
%\usetikzlibrary{positioning, calc, fadings, decorations.pathreplacing}
% Complex graphics with tikz, e.g. flow charts:
% \usetikzlibrary{backgrounds,shapes,calc,arrows} 
\usepackage{pgfplots} % Plots with pgf
\usepackage{pgfplotstable}
\pgfplotsset{compat=1.10} % make sure we are using a version of pgfplots that can handle the features we use!
\pgfplotsset{width=6cm} % orig: 5
\pgfplotsset{height=3.6cm} % orig: 3
\pgfplotsset{grid style={solid}}

% Graphics
% \usepackage[pdftex]{graphicx} % epstopdf support for graphicx
% \usepackage{epstopdf} % epstopdf package
\usepackage{graphicx} % package for loading graphics, e.g. png, pdf, eps, etc.
%\usepackage{subfigure}
%\usepackage{subfig}
% Moving away from subfig/subfigure, towards using
% the more recent caption/subcaption package which
% has more useful features and is quite easy to use:
\usepackage{caption}
%\captionsetup[figure]{labelfont=bf,%
                       %listformat=simple,%
                      %}
%\captionsetup[table]{labelfont=bf,
                      %listformat=simple,%
                      %position=below}
\captionsetup{%
  %format=hang,%
  indention=2em,%
  labelformat=default,%
  labelsep=colon,%
  font=onehalfspacing,%
  labelfont=bf,%
  listformat=simple,%
  position=below,%
  skip=1em,%
  hypcap=true%
}
\usepackage{subcaption}
\captionsetup[sub]{%
  indention=1em,%
  font+=small,%
  font+=onehalfspacing,%
  labelformat=parens,%
  labelsep=space,%
  subrefformat=parens,%
  skip=1ex,%
  list=false,%true,%
  hypcap=true%
}

% Colour
\usepackage{color} % package for color support
%\pagecolor{white} % setting pagecolor to white
%\usepackage{xcolor}
\usepackage{colortbl} % for colors in tables


% PDF packages
\usepackage[pdftex]{hyperref}
\hypersetup{pdfauthor=your name, pdftitle=PhD Thesis}
\hypersetup{colorlinks=true,
        linkcolor=black,
        citecolor=black,
        urlcolor=black}
\usepackage[final]{pdfpages}

% URL
\usepackage{url} % hyphenation for urls
%\usepackage[activate=normal]{pdfcprot} % moves the hyphen a bit over the right text-border when hyphenating a word.

% % Bibliography hyperref
\usepackage{natbib} % required for bibliography
% \bibliographystyle{plain}
\bibliographystyle{authordate1}
% \bibliographystyle{authyear}% \usepackage[authoryear,sort]{natbib} % required for bibliography

% to cite the title, e.g. \usebibentry{id}{title}
%\usepackage{keyval}
%\usepackage{usebib}
%\bibinput{references/books}

% \usepackage[style=authoryear,sorting=nyt]{biblatex}
% \bibliography{/home/frank/Documents/bibliography/image_processing}
% \usepackage[authoryear,sort]{natbib}
%\usepackage{natbib}
%\usepackage{multibib} % loading multiple bib files
%\citestyle{natdin}
% examples of using natbib:
% \citet{goossens93} Goossens et al. (1993)
% \citep{goossens93} (Goossens et al., 1993)
% \citet*{goossens93} Goossens, Mittlebach, and Samarin (1993)
% \citep*{goossens93} (Goossens, Mittlebach, and Samarin, 1993)
% \nocite{name} prints non-cited references
% \nocite{*} automatically prints all references in bib files


% Tabellen
\usepackage{booktabs}
\usepackage{array}
\usepackage{multirow}
%\usepackage{slashbox}
\usepackage{tabularx}
\usepackage{dcolumn}	% Decimal column alignments
\newcolumntype{d}[1]{D{.}{.}{#1}}


% Pseudocode
%\usepackage{algorithmic} % replaced by algpseudocode (compatible)
\usepackage[ruled,chapter]{algorithm} % plain, boxed, ruled
%\newcommand{\theHalgorithm}{\arabic{algorithm}}
%\numberwithin{algorithm}{chapter}
\usepackage{algpseudocode}
%\usepackage[noend]{algpseudocode}
\newcommand{\Commentinline}[1]{\hfill// \textit{#1}}
\renewcommand{\algorithmiccomment}[1]{// \textit{#1}}
\algnewcommand\algorithmicoutput{\textbf{Output:}}
\algnewcommand\Output{\item[\algorithmicoutput]}

\usepackage{blindtext}

% Change orientation of single pages:
\usepackage{pdflscape} % problem when printing, so only use it for online versions of the document
\usepackage{rotating} % e.g. \begin{sidewaystable}, or \begin{turn}{angle}

% Index
%\usepackage{makeidx}
%\makeindex % Index erzeugen


% my own clearpage till odd page
\newcommand\myclearpage{\clearpage\mypagestyle{empty} {\cleartooddpage}}
% my own clearpage till odd page
\newcommand\doubleclearpage{\cleartooddpage}


% labels/cross references for 'description' lists:
\usepackage{enumitem}
\makeatletter
\def\namedlabel#1#2{\begingroup
    #2%
    \def\@currentlabel{#2}%
    \phantomsection\label{#1}\endgroup
}
\makeatother

% footnotes with label/ref, e.g.:
% some text with a new footnote\fnmark{label}
% and on its own: \fntext{label}{footnote text}
\usepackage{refcount}
\newcommand{\fnmark}[1]{\refstepcounter{footnote}\label{#1}\footnotemark[\getrefnumber{#1}]}
\newcommand{\fntext}[2]{\footnotetext[\getrefnumber{#1}]{#2}}

%
% END
%
